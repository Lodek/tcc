\newcommand{\figdataparse}{
\begin{figure}
\begin{lstlisting}
type Symbol = Char

data Operator = Concat | Alternation deriving (Show, Eq)
data Quantifier = Kleene deriving (Show, Eq)
data Token = SToken Symbol | QtToken Quantifier | OpToken Operator | GroupBegin | GroupEnd deriving (Show, Eq)

data SubExpression = SubExp [Token] | QuantifiedSubExp [Token] Quantifier deriving (Show, Eq)

data ParseTree = Node ParseTree Operator ParseTree | QuantifierLeaf ParseTree Quantifier | Leaf Symbol deriving (Show, Eq)

\end{lstlisting}
\caption{Tipos de dados definidos para o modulo de parse.}
\label{f-data-parse}
\end{figure}
}

\newcommand{\figdataautomata}{
\begin{figure}
\begin{lstlisting}
data SigmaElem s = Symbol s | Epsilon deriving (Show, Eq)
type State = Int

type Sigma s = Set.Set (SigmaElem s)

type Delta s = (State -> SigmaElem s -> Set.Set State)

data Automata s = NFA { alphabet :: Set.Set (SigmaElem s),
                          states :: Set.Set State,
                          q0 :: State,
                          qas :: Set.Set State,
                          delta :: Delta s
                        }
\end{lstlisting}
\caption{Tipos de dados definidos para o modulo de automatas.}
\label{f-data-automata}
\end{figure}
}
