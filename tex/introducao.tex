\section{INTRODUÇÃO}
Qualquer estudante ou profissional que esteja envolvido no ambiente da tecnologia da informação certamente já teve contato com alguma espécie de linguagem de programação em algum ponto de sua jornada.
De acordo com uma pesquisa feita pelo StackOverflow, as 5 linguagens de programação mais usadas são: JavaScript, Python, Java, Linguagens de script (Bash, Powershell, Shell) e C\# \cite{stack-overflow}.

Embora esse conjunto possa parecer como um conjunto heterogêneo de tecnologias, divergindo fortemente em convençoes e nichos de uso, todas essas linguagens fazem parte da família de linguagens conhecidas como imperativas.
Inquestionavelmente, as liguagens imperativas são muito importantes, pois as mesmas compõem a maioria do código sendo produzido diariamente, porém não são a única família de linguagens existentes.
Nesse trabalho será discutido o paradigma de programação funcional, uma alternativa ao paradigma imperativo que domina o mercado.

O objetivo desse trabalho é introduzir o paradigma funcional ao leitor, começando por suas origens, introduzindo conceitos importantes e distinções com o paradigma imperativo.
A fim de mostrar exemplos de código, será implementado um motor de busca para expressões regulares em Haskell.
O código do motor de busca será usado de exemplo para ilustrar o paradigma funcional na prática.
Para isso, será introduzido de maneira superficial a teoria das automatas e expressões regulares.

Em conclusão, esse trabalho tem como objetivo introduzir o paradigma funcional, através da linguagem Haskell, comparando os paradigmas funcional e imperativo, e discutindo a maneira funcional de resolver certos problemas computacionais.
