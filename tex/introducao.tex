\section{INTRODUCAO}

%The intro
Qualquer estudante ou profissional que estja involvido no ambiente da tecnologia da informação certamente já teve contato com alguma espécie de linguagem de programação em algun ponto de sua jornada.
De acordo com uma pesquisa feita pelo StackOverflow, as 5 linguagens de programação mais usadas são: JavaScript, Python, Java, Linguagens de script (Bash, Powershell, Shell) e C# \cite{https://insights.stackoverflow.com/survey/2019#technology}.
Embora essa lista de linguagens possa parecer como um conjunto heterogêneo de tecnologia, divergindo fortemente entre conenções e nicho de uso, todas essas linguagens fazem parte da família de linguagens conhecidas como imperativas.
Inquestionavelmente, as liguagens imperativas são muito importantes, pois as mesmas compõem a maioria do código sendo produzido diariamente, porém não são a única família de linguagens existentes.
Nesse trabalho irá ser discutido o paradigma de programação funcional, uma alternativa ao paradigma imperativo que domina o mercado.


De maneira simplificada, pode-se dizer que programas em linguagens imperativas possuem um foco muito grande em definir o como um problema é resolvido.

\subsection{Objetivos}
Objetivo geral: Implementar uma biblioteca de processamento de expressoes regulares para a linguagem Haskell.

Objectivo especifico:
- Entender o paradigma funcional e os componentes da linguagem Haskell
- Estudar a teoria de automatas, variantes das expressoes regulares e algoritimos de processadores de expressoes regulares
- Implentar biblioteca de busca de Strings em Haskell

\subsection{Justificativa}
Embora o objetivo deste trabalho seja a construcao de uma biblioteca para processamento de expressões regulares, a motivacao para isso é de realizar um estudo sobre o paradigma funcional.

Embora ser muito utilizada, a programacao imperativa nao é o unico paradigma computacional existente.
Um dos paradigmas que contrasta diretamente é o paradigma de programacao funcional.
Em poucas palavras, a grande diferenca entre esses paradigmas é que enquanto a programacao imperativa foca no como resolver o problema, a programacao funcional foca no \emph{que} o programa faz.
O ponto de constrate sao as palavras como e que, embora seja uma definicao simplista, consegue capturar a essencia desses paradigmas.

Embora o paradigma funcional seja muito menos dissiminado, ele eh de extrema importancia.
Grandes descobertas nesse campo de estudo transcende o paradigma funcional e 'infecta' as linguagens procedurais.
Alguns exemplos são as list comprehenssions da linguagem Python e as lambda expressions e streams introduzidos na versão 8 da linguagem Java, um dos gigantes do mercado.
Sendo assim, esse trabalho tem como objetivo introduzir o paradigma funcional, focado na linguagem Haskell.
Serao introduzidos os principais conceitos e ferramentas das linguagems funcionais com um foco empirico.
Esses conceitos serao utilizados para implementar uma engine de processamento de texto usando expressoes regulares.

Em conclusão, o paradgima funcional faz parte do dia a dia da grande maioria dos profissionais da area, emboram muitas vezes nao saibam dissos.
Elementos de sucesso de linguagens funcionais acabam sendo implementados nas grandes linguagens imperativas pois os mesmos se mostraram uteis e atraente.
Sendo assim, desenvolvedores devem conhecer um minimo sobre o paradgima funcional, nao so pois ele esta presente em grande parte das linguages mas porque alguns problemas podem ser resolvidos de maneira mais breve.
