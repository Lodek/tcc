\section{INTRODUCAO}

%The intro
Qualquer estudante ou profissional que esteja envolvido no ambiente da tecnologia da informação certamente já teve contato com alguma espécie de linguagem de programação em algum ponto de sua jornada.
De acordo com uma pesquisa feita pelo StackOverflow, as 5 linguagens de programação mais usadas são: JavaScript, Python, Java, Linguagens de script (Bash, Powershell, Shell) e C# \cite{https://insights.stackoverflow.com/survey/2019#technology}.
Embora essa lista de linguagens possa parecer como um conjunto heterogêneo de tecnologias, divergindo fortemente em convençoes e nichos de uso, todas essas linguagens fazem parte da família de linguagens conhecidas como imperativas.
Inquestionavelmente, as liguagens imperativas são muito importantes, pois as mesmas compõem a maioria do código sendo produzido diariamente, porém não são a única família de linguagens existentes.
Nesse trabalho irá ser discutido o paradigma de programação funcional, uma alternativa ao paradigma imperativo que domina o mercado.

O foco desse trabalho será a criação de um módulo em Haskell para realizar buscas em textos usando expressões regulares.
Durante essa jornada serão feitas comparações entre algoritmos escritos de maneira imperativa e funcional, usando as linguagens Python e Haskell, respectivamente.
As expressões regulares partem da teoria da computação, mais especificamente da teoria das automatas.
Será definida a teoria as automatas, como elas são capazes de processar expressões regulares e finalemente será abordado a implementação do módulo em Haskell para a busca em texto.

Embora o paradigma funcional seja muito menos disseminado, ele é de extrema importância e possuem um grande impacto fora de seu nicho. as descobertas e inovações 
Diferentes inovações e descobertas no paradgima funcional, de certo modo infecta as linguagens imperativas.
Como exemplo disso temos que a partir da versão 8 do Java, foram introduzidas interfaces funcionais e \emph{arrow functions}, conceitos esses que surgiram a programação funcional.
Na linguagem Python, outro gigante da programação imperativa, existem as \emph{list comprehenssions}, uma maneira idiomática de se construir listas em Python.
Esse recurso muito amado da linguagem foi inspirado em um recurso muito similar que existe na linguagem Haskell.

Em conclusão, embora as linguagem funcionais sejam muito menos comum, elas definitivamente deixaram e continuam deixando marcas nos gigantes da programação.
Elas transcendem seu pequeno nicho de usuários e afetam a grande maiorias das pessoas que produzem código regularmente, mesmo que muitos não tenham ciência disso.
Sendo assim, esse trabalho tem como objetivo introduzir o paradigma funcional, focando na linguagem Haskell, realizando comparações entre os dois paradigmas e discutindo a maneira funcional de resolver certos problemas computacionais.

%De maneira simplificada, pode-se dizer que programas em linguagens imperativas possuem um foco muito grande em definir o como um problema é resolvido.
%Em poucas palavras, a grande diferenca entre esses paradigmas é que enquanto a programacao imperativa foca no como resolver o problema, a programacao funcional foca no \emph{que} o programa faz.
