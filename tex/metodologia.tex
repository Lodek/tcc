\section{METODOLOGIA}
Nessa seção será abordada a arquitetura do software desenvolvido, isso permite uma visão holistica que define sua estrutura.
O software desenvolvido é uma biblioteca de busca usando expressões regulares.
Essa biblioteca foi quebrada em três modulos privados e um público.
Os modulo internos definem os tipos de dados e implementam as transformações, enquanto o modulo público serve como uma interface que conecta os modulos.
Os quatro modulos são: modulo de parse, modulo de automata, modulo de conversão e modulo publico.

\subsection{Modulo de Parse}

O modulo de parse é o primeiro estágio da pipeline que irá permitir a construção de uma automata de busca.
Esse módulo é responsável por converter a entrada do usuário (uma String) em uma estrutura intermediária que é consumida pelo módulo conversor.

A saída do parser é uma estrutura em árvore, similar à uma árvore de parse gerada por um compilador.
Nessa árvore, suas folhas são as primitivas do alfabeto de entrada (letras como “a”, “b” ou “c”), e os nós se interligam através dos operadores de concatenação e alteração.
Existe uma exceção para os nós pois eles podem representar uma quantificação também.
Logo, um nó é uma operação que liga subárvores ou é uma quantificação que permite o uso do operador “*”.

A escolha de uma árvore para essa estrutura intermediária é conveniente por dois motivos. 
Primeiro, subárvores apresentam uma tradução, quase que, direta com automatas primitivas.
Segundo, o uso da árvore elimina ambiguidades referentes à ordem das operações, sem precisar fazer uso de parênteses para indicar a qual grupo de caracteres um operador pertence.

\begin{figure}
\begin{lstlisting}
type Symbol = Char

data Operator = Concat | Alternation
data Quantifier = Kleene
data Token = SToken Symbol 
             | QtToken Quantifier 
             | OpToken Operator 
             | GroupBegin 
             | GroupEnd

data SubExpression = SubExp [Token] 
                     | QuantifiedSubExp [Token] Quantifier

data ParseTree = Node ParseTree Operator ParseTree 
                 | QuantifierLeaf ParseTree Quantifier 
                 | Leaf Symbol

\end{lstlisting}
\caption{Tipos de dados definidos para o modulo de parse.}
\label{f-data-parse}
\end{figure}



Para construir a árvore, o modulo define alguns tipos de dados, na fígura \ref{f-data-parse} são dadas suas definições.
O típo \emph{Symbol} é sinomimo de um tipo de caractere.
O tipo \emph{Operator} define uma enumeração, podendo ser ou uma concatenação ou uma alternação.
O tipo \emph{Quantifier} define uma enumeração, representando o operador de Kleene, também conhecido como estrela.
O tipo \emph{Token} define um grupo de construções, podendo ser um token simbolico, token de quantificação, token de operação, delimitador de inicio de grupo ou delimitador de fim de grupo.
O tipo \emph{SubEpression} representa uma sub-expressão composta por uma lista de tokens ou uma subexpressão quantificada.
Finalmente, o tipo \emph{ParseTree} representa uma árvore, podendo ter uma folha contendo um símbolo; um nó contendo uma arvore e uma quantificação; um nó contendo uma árvore, operador e outra árvore.



A implementação desse módulo em Haskell foi feita usando um conjunto de funções que opera sobre os tipos definidos anteriormente.
A tabela \ref{t-parse-funcs} indica as funções desse módulo, junto com seus tipos de entrada, saída e uma breve descrição sobre cada uma.

\begin{table}
  \begin{tabular}{lll}
  \hline
  Nome & Assinatura & Descrição \\
  \hline
  genToken & Symbol -> Token & Transforma um símbolo em um token. \\
  genTokens & [Symbol] -> [Token] & Transforma uma lista de símbolos em uma lista de tokens. \\
  normalizeStream & [Token] -> [Token] & Adiciona operador de concatenação entre símbolos justapostos. \\
  evenGroupPredicate & [Token] -> Bool & Valida que existe numero par de parenteses em uma regex. \\
  uniqueQuantifierPredicate & [Token] -> Bool & Valida que não existem quantificadores justapostos. \\
  yankSubExp & [Token] -> (SubExpression, [Token]) & Extrai uma subexpressão do stream de tokens. \\
  takeWhileGroupUneven & [Token] -> [Token] & Remove tokens até que um grupo completo seja formado. \\
  takeWhileList & ([Token], Bool) -> [Token] -> [Token] -> [Token] & Remove items da primeira lista e adiciona a segunda enquanto o predicado é verdadeiro.\\
  validateTokens  & [Token] -> Bool & Valida tokens \\
  sortAndTreefy & [Token] -> [Either Operator ParseTree] & Classifica Token como sendo um Operador ou uma árvore. \\
  buildSupExp & SubExpression -> ParseTree & Transforma uma subexpressão em uma arvore. \\
  transformEithers & [Either Operator ParseTree] -> ([Operator], [ParseTree]) & Agrupa operadores e árvores. \\
  mergeOps & [Operator] -> [ParseTree] -> [ParseTree] & Combina árvores usando uma lista de operadores. \\
  buildTree & String -> ParseTree & Transforma um string em uma árvore. \\
 \hline
  \end{tabular}
\caption{Tabela de funções para o modulo de parse. Cada função é apresentada com sua assinatura e uma breve descrição.}
\label{t-parse-funcs}
\end{table}


No módulo é definida a função \emph{buildTree} que recebe uma String e retorna uma árvore, uma função que faz todo o processo de conversão.
Essa função funciona como uma pipeline de transformações sobre a entrada, até gerar a saída final.
Primeiramente \emph{buildTree} recebe um valor do tipo string o transforma para uma lista de tokens, que passam por um processo de normalização.
O processo de normalização adiciona operadores de concatenação entre simbolos, visto que concatenação é uma operação implicita na regex.
Após normalizada, a regex é validada usando uma lista  de predicados, caso a regex esteja ok, a computação prossegue, caso não o programa retorna um erro.
Os tokens validados são convertidos para um par de listas: os simbolos são convertidos para uma lista de folhas e os operadores são adicionados à lista de operadores.
Por fim, as folhas são combinadas usando os operadores da lista de operadores, o resultado do processo de união é uma árvore onde os nós são as folhas unidas pelos operadores.

O fluxo acima descreve o processo de processamento para uma expressão regular sem sub-grupos.
No caso de um subgrupo, o mesmo processo é seguido, porém de maneira recurssiva.
Ao encontrar um grupo o algoritimo de agrupamento de tokens chama a pipeline de conversão novamente.
No momento em que não existem mais subgrupos, o algoritimo retorna com uma árvore e o processamento pode continuar.

\subsection{Modulo de Automatas}

O modulo de automatas implementa funções que permite simular o funcionamento de uma automata.
Esse modulo define um tipo de dado que representa uma maquina de estados e expõe funções que operam sobre uma maquina de estado.

Primeiramente, foram definidos os tipos de dados necessarios para modelar a maquina de estado.
A fígura \ref{f-data-automata} contém o trecho de código que define os tipos de dados.
Foi definido o tipo \emph{SigmaElem} que representa um elemento do alfabeto da automata, esse tipo contém o construtor \emph{epsilon} que indica o símbolo de entrada nulo.
O tipo \emph{Sigma} representa o alfabeto da automata, ou seja conjunto de \emph{SigmaElem}.
O tipo \emph{State} é um alias para um numero inteiro.
O tipo Delta é um alias para função de transição da maquina de estado, igual ao definido na literatura.
O tipo NFA define uma maquina de estados.
NFA é uma tupla de cinco elementos: um alfabeto de tipo \emph{Sigma}, um conjunto de estados, o estado inicial, conjunto de estados de aceitação e a função de transição.
Percebe-se que a modelagem de uma automata segue exatamente o modelo definido na literatura.

\begin{figure}
\begin{lstlisting}
data SigmaElem s = Symbol s | Epsilon deriving (Show, Eq)
type State = Int

type Sigma s = Set.Set (SigmaElem s)

type Delta s = (State -> SigmaElem s -> Set.Set State)

data Automata s = NFA { alphabet :: Set.Set (SigmaElem s),
                          states :: Set.Set State,
                          q0 :: State,
                          qas :: Set.Set State,
                          delta :: Delta s
                        }
\end{lstlisting}
\caption{Tipos de dados definidos para o modulo de automatas.}
\label{f-data-automata}
\end{figure}

\begin{table}
  \begin{tabular}{lll}
  \hline
  Nome & Assinatura & Descrição \\
  \hline
  buildSigma & [a] -> Set (SigmaElem a) &  Constroi um alfabeto a partir de uma lista. \\
  unionAndApplyOverDiff & (Set a -> Set a) -> Set a -> Set a -> (Set a, Set a) & Retorna a união dos conjuntos e o resultado de uma função aplicada sobre a diferença de dois conjuntos. \\
  deltarOverSet & Delta s -> SigmaElem s -> Set State -> Set State & Aplica delta sobre simbolos, retorna todos os estados acessiveis
  epsilonClosure & Delta s -> Set State -> Set State & Retorna todos estados acessiveis usando a transição nula.\\
  stateEpsilonClosure & Delta s -> State -> Set State & Retorna estados acessiveis usando a transição nula para um uninco estado.
  move & Delta s -> State -> SigmaElem -> Set State & Dado um símbolo e um estado, retorna todos estados acessiveis. \\
  eval & Delta s -> State -> Set State -> [s] -> Bool & Aplica a maquina de estados sobre um stream de entrada
 \hline
  \end{tabular}
\caption{Tabela de funções para o modulo de parse. Cada função é apresentada com sua assinatura e uma breve descrição.}
\label{t-automata-funcs}
\end{table}


As funções definidas nesse módulo são apresentadas na tabela \ref{t-automata-funcs}.
A implementação da automata foi feita de acordo com a literatura \cite{dragon-book}, onde é apresentado os algoritimos para simular uma automata.
O algoritimo apresentado define algumas funções auxiliares, tal como \emph{epsilonClosure} e \emph{move}.
As outras funções do modulo foram introduzidas devido a conversão do algoritimo imperativo da literatura para um algoritimo funcional.

Para fazer uso do maquinario que simula uma automata é necessario declarar uma NFA.
A simulação de uma computaçao é feita usando a função \emph{eval} que recebe uma maquina de estados,  uma lista de símbolos e retorna um valor booleano.
Após processar todos os símbolos, caso a maquina de estado esteja em um estado de aceitação o retorno da função é verdadeiro, caso contrario falso.

Usando a arquitetura apresentada, a implementação das funções definidas permite a simulação de uma maquina de estados.
Essa maquina de estados sera utilizada para executar a busca pela expressão regular.
Para isso, é necessário converter a regex em árvore em uma maquina de estados.

\subsection{Modulo de conversão}

O modulo de conversão é responsável por converter a expressão regular armazenada em uma árvore para uma automata que possa ser executada.

A literatura contém maquinas de estados para as primitivas de uma expressão regular.
Usando essas primitivas, é possível combina-las para construir expressões mais complexas.
O módulo de conversão disponibiliza funções para criar e combinar essas primitivas.
A partir dessas funções, basta trafegar pela árvore para transforma-la em um unico valor, ou seja, usar um \emph{fold}.

\begin{table}
  \begin{tabular}{lll}
  \hline
  Nome & Assinatura & Descrição \\
  \hline
  singletonNFA & SigmaElem s -> Automata s & Gera uma automata para um caractere. \\
  singletonDelta & SigmaElem s -> State -> State -> Delta s & Gera a função de transição para um caractere. \\
  wrapState & State -> SigmaElem s -> Set State -> Delta s -> Delta s & Função auxiliar para adicionar transições. \\
  addState & State -> Delta s -> Delta s & Adiciona um novo estado à função de transição. \\
  addStateWithTransitions & State -> [(SigmaElem s, Set State)] -> Deta s -> Delta s & Adiciona um estado com transições. \\
  alternateDelta &  State -> State -> NFA s -> NFA s -> NFA s & Combina duas automatas tal que o resultado seja equivalente a operação de alternação em uma regex. \\
  concatNFA & NFA s -> NFA s -> NFA s & Combina duas automatas tal que o resultado seja equivalente a operação de concatenação em uma regex. \\
  kleenifyNFA & State -> State -> NFA s -> NFA s & Adiciona estados à uma automata tal que o resultado seja equivalente ao operador "*" em uma regex.\\
  foldParseTree & ParseTree -> NFA s & Converte a arvore de parse em uma automata.\\
 \hline
  \end{tabular}
\caption{Tabela de funções para o modulo de conversão. Cada função é apresentada com sua assinatura e uma breve descrição.}
\label{t-translator-funcs}
\end{table}


A tabela \ref{t-translator-funcs} apresenta as funções definidas para o módulo.
Note que esse modulo não introduz um tipo de dado próprio, apenas realiza transformações.

\subsection{Modulo público}
O módulo público cria uma interface para que a biblioteca seja facil de utilizar.
Nesse módulo é definida uma única função, \emph{match}, que recebe duas strings: uma expressão regular e um texto sobre o qual a busca é feita.
\emph{Match} apenas faz chamadas para as funções definidas nos outros módulos e não introduz nenhuma logica nova.
Esse módulo é apenas uma interface para o usuário.

\subsection{Práticas comuns e observações}
Embora os módulos da biblioteca tenham objetivos distintos, algumas práticas foram utilizadas em todos eles.
Para o desenvolvimento das funções foi feito o uso de uma metodologia usando testes untiarios, em que para cada função é escrito um teste para validar seu comportamento.
Para isso, foi utilizado a biblioteca HUnit do Haskell, uma biblioteca que auxilia na execução de casos de testes.
A prática de desenvolver testes para as funções dos módulos foi de extrema importância pois permite validar cada unidade lógica do código de maneira individual.

Nessa seção foram introduzidos os diferentes módulos que compõe a biblioteca proposta: parse, automata, conversão e publico; também foi visto as funções propostas para cada módulo.
Essa descrição da arquitetura do software permite o seu entendimento sem entrar nas nuancias associadas a implementação das funções.
Além disso, permite que diferentes implementações existam, pois uma vez definida a interface do programa, basta que as partes previstas existam para que o mesmo funcione, sem existir a dependencia de como a interface é implementada.
O código completo para a biblioteca está disponivel no apendice 1 e no site http://github.com/lodek/regex-engine.
