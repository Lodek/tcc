\subsection{METODOLOGIA}

Como explicado na introdução, o foco destre trabalho é demonstrar alguns  elementos da programação funcional.
Para isso, foi escolhido o problema de implementar um módulo de busca de strings usando expressões regulares.
Será escolhido trechos de código especialmente interessante do módulo escrito que serão explciados a fundo.

Foi visto que uma expressão regular pode ser convertida em uma automata equivalente.
Sendo assim, o problema possui duas tarefas: criar submódulo para converter uma regex em uma automata e implementar um submódulo que permita criar e operar uma automata.
Serão definidas as arquiteturas de cada submódulo tal como o encadeamento de funções que serão chamadas para resolver cada problema, análogo ao que foi feito anteriormente.
Será explicado, de maneira alto nível, o que cada função faz baseada em suas entradas e saídas.
Isso irá motivar a introdução de tipos de dados únicos a programação funcional.

Além dessa inspeção de "caixa preta" das funções, os pontos principais do módulo será explicado em detalhe, o que permitirá aa análise de conceitos importantes no paradigma funcional.
Será feita uma comparação entre trechos escrito de maneira funcional e imperativa.
Essa comparação tem dois objetivos: introduzir conceitos referentes a linguagem funcional e identificar em quais situações um código funcional é mais simples, ou mais complexo, que o seu equivalente de maneira imperativa.
Para introduzir os conceitos do paradigma funcional, o código irá ser projetado tal que demonstre as diferentes ferramentas que compõe a caixa de ferramentas de um programador funcional.
As ferramentas simples abordarão conceitos comos imutalidade e recursão e as ferramentas mais complexas irão introduzir abstrações muito perculiares da programação funcional tal como \emph{Functors} e \emph{Applicative Functors}.
A metodologia escolhida tem como objetivo ser transparente quanto aos lados bons e ruins da programação funcional e também auxiliar a associação do paradigma imperativo ao funcional.
Dessa forma, um leitor familiar com programação imperativa poderá entender como um problema resolvido de maneira imperativa pode ser traduzido para um algoritimo funcional.

Em conclusão, o trabalho irá resolver o problema de criar um módulo de procura em texto usando expressões regulares.
O problema será quebrado em funções, exemplificando como resolver um problema a partir de funções ao invez de passos.
O código fonte do módulo criado será usado para introduzir conceitos sobre o paradigma funcional e familiarizar o leitor com algumas ferramentas.
Ao mesmo tempo, trechos de códigos funcionais serão comparados com seu equivalente escrito em uma linguagem imperativa, o que permitira associar conceitos imperativos a funcionais e expor os pontos fortes e fracos desse paradigma.
