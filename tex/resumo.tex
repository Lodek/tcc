O objetivo deste trabalho é introduzir programação funcional mencionando sua origem e alguns de seus conceitos principais.
Alguns do conceitos explorados são: imutabilidade, funções como cidadãos de primeira classe, funções de ordem superior e recursão.
A fim de demonstrar um pouco sobre o pensamento e processo de desenvolvimento funcional é implementado, na linguagem Haskell, um motor de busca que usa expressões regulares.
Para contextualizar o problema é introduzido os básicos sobre expressões regulares e automatas.
O artigo introduz a arquitetura projetada na implementação do motor de busca e aprofunda em partes do código a fim de demonstrar como os principais conceitos da programação funcional foram usados na prática.
Palavras-chaves: Programação funcional, Expressões Regulares, Automatas
