\subsection{Programação Funcional}

%A maquina de Turing foi muito importante para a evolução dos computadores, um modelo idealizado capaz de resolver vários problemas a partir de instruções simples.
%Usando o modelo da maquina de Turing, os processadores foram projetados para mimicar essas operações, introduzindo instruções leitura, comparação e jump.
%Como consequencia, vieram as linguagens assembly, uma especie de dicionario que permite programar usando palavras ao invés de códigos de instruções.
%Embora assembly tenha facilitado muito a programação, ela ainda é muito próximo do hardware e extremamente trabalhoso de definir os passos para escrever um programa.
%Para resolver isso veio a programação estruturada, que introduziu os ifs, whiles e fors.
%O processo de design de linguagens foi longo, foram diversas abstrações, construidas de maneira iterativa.
%Atualmente o paradigma considerado mais alto nível e muito usado é a orientação a objetos.
%Perceba que as linguagens imperativas foram construídas a partir de uma construção simples e que o resultado foi um processo longo para tornar esse modelo mais intuitivo aos humanos, especialmente com a programação orientada a objetos.

Programação funcional é um paradgima computacional que, de certa forma, contrasta com o paradigma imperativo.
Esse paradigma é um topico extenso e rico, com uma longa historía por traz.
Resumir esse assunto amplo é um desafio pois várias foram as contribuições e descobertas nesse campo de estudo.
Um ponto de inicio é a definição dada por Bird \cite{Bird},
\begin{quotation}
"Programção funcional é: um método para construção de programas que enfatiza funções e suas aplicações ao invéz de commandos e suas execuções; programação funcional faz uso de notações matemática simples que permite que problemas sejam descritos de maneira clara e concisa. [...]".
\end{quotation}
Esse paradigma difere do imperativo pois a programação imperativa foca em passos para resolver um problema.
O paradigma funcional tira o foco dos passos individuais para solucionar o problema e enfatiza uma estrutura para resolver o problema.

Embora seja dificil definir exatamente o paradigma funcional, a sua origem é bem clara.
De maneira simplificada, o paradigma funcional veio a partir de um modelo computacional conhecido como Calculo Lambda.
Segundo \cite{lambda}, o calculo lambda é um modelo de computabilidade criado por Alonzo Church em 1930
Nesse modelo, a operação basica é a aplicação de funções \cite{lambda}.
O calculo lambda teve um impacto muito importante na programação funcional, e alguns autores defendem que o calculo lambda é fundamental para a aprendizagem de programação funcional, porém nesse trabalho esse tópico não será abordado.

A grande linguagem lisp marca as origens da programação funcional pois foi a primeira linguagem baseada no calculo lambda \cite{graham}.
Essa seria apenas a primeira de um grande numero de linguagens que se baseariam nesse modelo.

Deseja-se ressaltar que a famosa maquina de Turing, um conceito muito famoso da teoria da computabilidade, é equivalente ao calculo lambda.
Embora a maquina de Turing e o calculo lambda seja ideologicamente diferentes, foi comprovado que os dois são equivalentes, hipotese essa conhecida como a hipotese de Churchill-Turing \cite{computability}.
Isso significa que os problemas que podem ser resolvidos por uma maquina de Turing, e como consequencia um computador moderno, podem ser resolvidos usando uma linguagem funcional.

Em seguida serão abordados aspectos mais técnicos da programação funcional.

\subsubsection{Imutabilidade}

Um conceito comumente encontrado na programação funcional é a imutabilidade.
Uma linguagem imutavel trata as variaveis de um programa de maneira similar à matemática, tal que o valor de uma variavel so pode ser definido no momento de sua inicializacão. 
Isso é equivalente a definir todas as variaveis como final ou constante, dependendo da linguagem imperativa.

A imutabilidade é desejavel pois permite racicionar sobre o programa de maneira mais facil, pois sabemos com certeza que um valor não ira mudar apos ter sido inicializado.
Isso serve como uma especie de invariante que permite racionalizar sobre o programa, já que o programa não contém nenhum estado que varia com o tempo.
Isso evita diversos problemas comuns que ocorre quando compartilhamos objetos, desde falta de atenção pela parte do programador até condições de corrida impostas pelo algoritimo.

Essa restrição é interessante pois altera muito a maneira como algoritimos são escritos.
A ausencia da mutabilidade implica que não existem variaveis acumuladoras nem contadores.
Sem contadores, a alternativa para repetir um bloco de código por n vezes passa a ser a recursão.
A recursão é extremamente utilizada na programação funcional pois ela permite que uma funcão realize uma computação repetitiva, sem mutar valores.

\subsubsection{Funções como um cidadão de primeira classe}

Na programação, os tipos primitivos de uma linguagem são os blocos a partir do qual é possível construir estruturas complexas.
Os tipos primitivos podem ser armazenados em uma variável, passados para uma função, e normalmente existem operadores para esses tipos.
Na a programação funcional, uma função é um tipo de dado primitivo, isso significa que é possível declarar e armazenar uma função em uma variavel, passar uma função para uma função e receber uma função como retorno de uma função \cite{whyfpm}.

Na programação funcional, usar uma função como um tipo de dado é uma pratica essencial para criar abstrações.
Uma função que recebe uma função como argumento é chamada de função de ordem superior.
Hughes \cite{whyfpm} argumenta que as funções de ordem superior são essenciais pois elas permitem uma melhor reusabilidade de código.
Essa pratica é tão comum e poderoza que diversas linguagens populares, tal como JavaScript e Python, possuem esse tipo de função no seu core, tais como as funções: \emph{map}, \emph{filter} e \emph{reduce}.
%In video, explain what each functino does

Um exemplo dessa abstração é a função \emph{reduce}, ou como é chamada em Haskell, \emph{fold}.
Essa função é normalmente utilizada para iterar sobre uma lista de valores e produzir um novo valor.
Porém, essa abstração em especifico é extremamente poderoza, em \cite{graham} o autor argumenta a favor de sua expressividade.

As três funções mencionadas são exemplos de funções de ordem superior reutilizaveis e expressivas.
É interessante notar que uma função de ordem superior, muitas vezes, pode ser extendida para aceitar diferentes tipos.
Em Haskell, existe varios tipos de dados que aceitam a função \emph{fold}, não so listas, e em \cite{whyfpm} é argumentado que cada tipo de dado definido deve tambem implementar funções de ordem superior a fim de operar sobre esse tipo.

%For video, talk about functions that return functions
%Adding behavior before/after method, python decorators and aspects

Em conclusão, funções como um cidadão de primeira classe permite tratar funções como valores e realizar tranformações com elassobre elas de maneira transparente.
Esse conceito permite que funções de ordem superior existam na linguagem e foi argumentado a favor do poder de abstração dessa prática.

\subsubsection{Indo além}

Existem vários outros importantes conceitos sobre programação funcional, porém por motivos de breviedade eles não serão comentados nesse artigo mas sim, mencionados e direcionados para outras literaturas.

Um tópico polarizador em programação funcional é o assunto de \emph{laziness} e \emph{eager}, referindo a quando um valor será computado.
Existem vantagens e desvantagens para ambos; laziness é interessante por melhor performance em alguns casos, porém dificulta raciocinar sobre o programa.
É importante mencionar que em \cite{whyfpm}, o autor argumenta que laziness é fundamental para abstrair programas funcionais.

Outro ponto importante é sobre tipos de dados algebricos.
Tipos de dados algebricos permitem a implementação de tipos de dados recurssivos.
Em Haskell, uma lista é um tipo de dado recurssivo, com dois construtores.
Tipos de dados algebricos possibilitam \emph{pattern matching}, uma maneira sucinta de verificar a estrutura de um dado.
Uma boa introdução pode ser encontrada em \cite{lipovaca}.

Por último, é importante mencionar alguns assuntos que surgiram no Haskell, dentre eles type classes e monads.
Type classes foi a solução implementada em Haskell para um problema muito comum em linguagens de programação: override operadores \cite{haskell-ivory}.
Monads é praticamente uma buzz-word em programação funcional, especialmente na comunidade Haskell.
Sobre monads, é interessante mencionar que eles foram a solução para um grande problema que Haskell teve: como ser uma linguagem pura porém com efeitos colaterais \cite{haskell-ivory}.

\subsubsection{Conclusão}

Programação funcional é um tópico extenso com uma historia rica e de maneira nenhuma seria possível introduzir tudo nesse artigo.
Foi visto como as origens das linguagens funcionais diferem das linguagens imperativas, sendo baseadas no calculo lambda.
Em seguida foi introduzido dois conceitos importantes: imutabilidade e função como um cidadão de primeira classe.
Foi argumentado a favor da modularidade e abstração que esses conceitos introduzem.
Por fim, foi comentado sobre diferentes conceitos importantes para as linguagens funcionais, porém que fogem do escopo desse trabalho.

\section{METODOLOGIA}

%we doing it live 
%tools used to generate the data
%what is needed to reporduce the work

Nessa seção será abordada a arquitetura do software desenvolvida, isso permite uma visão holistica que define uma estrutura.
O software desenvolvido é uma biblioteca para busca usando expressão regular.
Essa biblioteca foi quebrada em três modulos publicos e um privado.
Os modulo internos definem os tipos de dados e implementam as transformações, enquanto o modulo público serfve como uma interface que conecta os modulos e uma interface de entrada / saida.
Os quatro modulos são: modulo de parse, modulo de automata, modulo de conversão e modulo publico.

\subsection{Modulo de Parse}

O modulo de parse é o primeiro estágio da pipeline que irá permitir a construção de uma automata de busca.
Esse módulo é responsável por converter a entrada do usuário (uma String) em uma estrutura intermediária que é consumida pelo módulo conversor.

A saída do parser léxico é uma estrutura em árvore, similar à uma árvore de parse gerada por um compilador.
Nessa árvore, as folhas dela são as primitivas do alfabeto de entrada (letras como “a”, “b” ou “c”), e os nós se interligam através de operadores, como os operadores de concatenação e alteração da regex.
Existe uma exceção para os nós pois eles podem representar uma quantificação também.
Logo, um nó ou é uma operação que liga subárvores ou é uma quantificação que permite o uso do operador “*”, por exemplo.

A escolha de uma árvore para essa estrutura intermediária é conveniente por dois motivos: subárvores apresentam uma tradução, quase que, direta com as automatas primitivas que equivalem a expressões regulares e o uso da árvore elimina as ambiguidades referentes à ordem das operações, sem precisar fazer uso de parênteses para indicar a qual grupo de caracteres um operador opera sobre.

Para construir a árvore, o modulo define alguns tipos de dados.
Na fígura \ref{f-parse-data} é dada a definição dos tipos de dados definidos.
O típo \emph{Symbol} é sinomimo de um tipo de caractere.
O tipo \emph{Operator} define uma enumeração, podendo ser ou uma concatenação ou uma alternação.
O tipo \emph{Quantifier} define uma enumeração, representando o operador de Kleene, também conhecido como estrela.
O tipo \emph{Token} define um grupo de construções, podendo ser um Token simbolico, token de quantificação, token de operação, delimitador de inicio de grupo ou delimitador de fim de grupo.
O tipo \emph{SubEpression} representa uma sub-expressão composta por uma lista de tokens ou uma subexpressão quantificada.
Finalmente, o tipo \emph{ParseTree} representa uma árvore, podendo ter uma folha contendo um símbolo; um nó contendo uma arvore e uma quantificação; um nó contendo uma árvore, operador e outra árvore.


A implementação desse módulo em Haskell foi feita usando um conjunto de funções que opera sobre os tipos definidos anteriormente.
A tabela \ref{t-parse-funcs} indica as funções desse módulo, junto com seus tipos de entrada, saída e uma breve descrição sobre cada uma.
Note que a função buildTree recebe uma String e retorna uma árvore, sendo assim essa função realiza a transformação completa sobre a entrada.

\begin{table}
  \begin{tabular}{lll}
  \hline
  Nome & Assinatura & Descrição \\
  \hline
  genToken & Symbol -> Token & Transforma um símbolo em um token. \\
  genTokens & [Symbol] -> [Token] & Transforma uma lista de símbolos em uma lista de tokens. \\
  normalizeStream & [Token] -> [Token] & Adiciona operador de concatenação entre símbolos justapostos. \\
  evenGroupPredicate & [Token] -> Bool & Valida que existe numero par de parenteses em uma regex. \\
  uniqueQuantifierPredicate & [Token] -> Bool & Valida que não existem quantificadores justapostos. \\
  yankSubExp & [Token] -> (SubExpression, [Token]) & Extrai uma subexpressão do stream de tokens. \\
  takeWhileGroupUneven & [Token] -> [Token] & Remove tokens até que um grupo completo seja formado. \\
  takeWhileList & ([Token], Bool) -> [Token] -> [Token] -> [Token] & Remove items da primeira lista e adiciona a segunda enquanto o predicado é verdadeiro.\\
  validateTokens  & [Token] -> Bool & Valida tokens \\
  sortAndTreefy & [Token] -> [Either Operator ParseTree] & Classifica Token como sendo um Operador ou uma árvore. \\
  buildSupExp & SubExpression -> ParseTree & Transforma uma subexpressão em uma arvore. \\
  transformEithers & [Either Operator ParseTree] -> ([Operator], [ParseTree]) & Agrupa operadores e árvores. \\
  mergeOps & [Operator] -> [ParseTree] -> [ParseTree] & Combina árvores usando uma lista de operadores. \\
  buildTree & String -> ParseTree & Transforma um string em uma árvore. \\
 \hline
  \end{tabular}
\caption{Tabela de funções para o modulo de parse. Cada função é apresentada com sua assinatura e uma breve descrição.}
\label{t-parse-funcs}
\end{table}


A arquitetura do módulo de parse é dada pelas funções na tabela \ref{t-parse-funcs} e os tipos de dados enumerados no texto.
A partir desses elementos, é possível implementar as funções tal que o modulo receba uma expressão regular em uma string e retorne uma árvore que preserve o significado semantico da expressão regular de entrada.

\subsection{Modulo de Automata}

O modulo de Automata implementa o modelo computacional das maquinas de estados.
Esse modulo define um tipo de dado que representa uma maquina de estados e expõe funções que operam sobre uma maquina de estado.

Primeiramente, foram definidos os tipos de dados necessarios para modelar a maquina de estado.
A fígura \ref{f-automata-data} contém o trecho de código que define os tipos de dados.
Foi definido o tipo \emph{SigmaElem} que representa um elemento do alfabeto da automata.
O tipo \emph{Sigma} o alfabeto da automata, ou seja conjunto de \emph{SigmaElem}.
O tipo \emph{State} é um alias para um numero inteiro.
O tipo Delta é um alias para função de transição da maquina de estado, igual ao definido na literatura.
Finalmente, analogo à literatura, uma maquina de estados é uma tupla de cinco elementos: um alfabeto de tipo \emph{Sigma}, um conjunto de estados, o estado inicial, conjunto de estados de aceitação e a função de transição.
Percebe-se que a modelagem de uma automata segue exatamente o modelo definido na literatura.

As funções definidas nesse módulo são apresentadas na tabela \ref{t-automata-funcs}.
A implementação da automata foi feita de acordo com a literatura \cite{dragon-book}, onde é apresentado os algoritimos para simular uma automata.
O algoritimo apresentado define algumas funções auxiliares, tal como \emph{epsilonClosure} e \emph{move}.
As outras funções do modulo foram introduzidas devido a conversão do algoritimo imperativo da literatura para um algoritimo funcional.

Usando a arquitetura apresentada, a implementação das funções definidas permite a simulação de uma maquina de estados.
Essa maquina de estados sera utilizada para executar a busca pela expressão regular.
Para isso, é necessário converter a regex em árvore em uma maquina de estados.

\subsection{Modulo de conversão}

O modulo de conversão é responsável por converter a expressão regular armazenada em uma árvore em uma automata que possa ser executada.

A literatura contém maquinas de estados para as primitivas de ume expressão regular.
Usando essas primitivas, é possível combina-las para construir expressões mais complexas.
O módulo de conversão faz disponibiliza funções para criar e combinar essas primitivas.
A partir dessas funções, basta trafegar pela árvore para transforma-la em um unico valor, ou seja, usar um \emph{fold}.

A tabela \ref{t-translator-funcs} apresenta as funções definidas para o módulo.
Note que esse modulo não introduz um tipo de dado próprio, apenas realiza transformações.

\subsection{Modulo público e comando cli}
O módulo público cria uma interface para que a biblioteca seja facil de utilizar.
Nesse módulo são definidas funções auxiliares e também um programa que pode ser rodado de maneira idependente.

As funções do modulo público são apresentadas na tabela \ref{t-wrapper-funcs}.
Essas funções apenas fazem chamadas para as funções dos diferentes modulos da biblioteca e não introduz nenhuma logica nova.

O comando CLI é diferente pois ele deve tratar com a entrada e saída de valores, ou seja, realizar operações com o sistema.
O programa criado recebe uma expressão regular como um argument do programa e le um texto para receber realizar a busca no stream de entrada padrão (stdin) e retorna o trecho do texto que satisfaz a regex.

\subsection{Práticas comuns e conclusão}
Embora os módulos da biblioteca tenham objetivos distintos, algumas práticas foram utilizadas para implementar todos eles.
Para o desenvolvimento das funções foi feito o uso de uma metodologia usando testes untiarios, em que para cada função é escrito um teste para validar seu comportamento.
Para isso, foi utilizado a biblioteca HUnit do Haskell, uma biblioteca que auxilia na execução de casos de testes.
A prática de desenvolver testes para as funções dos módulos foi de extrema importância pois permite validar cada unidade lógica do código de maneira individual.

Nessa seção foram introduzidos os diferentes módulos que compõe a biblioteca proposta: parse, automata, conversão e publico; também foi visto as as funções propostas para cada módulo.
Essa descrição da arquitetura do software permite o seu entendimento sem entrar nas nuancias associadas a implementação das funções.
Além disso, permite que diferentes implementações existam, pois uma vez definida a interface do programa, basta que as partes previstas existam para que o mesmo funcione, sem existir a dependencia de como a interface é implementada.
O código completo para a biblioteca está disponivel no apendice 1 e no site http://github.com/lodek/regex-engine.

