\subsection{Expressões Regulares}
As expressões regulares foram escolhidas como o problema computacional de interesse para introduzir as expressões regulares, porém como elas não são o foco destre trabalho, sua abordagem será simplificada.
Nessa secão será introduzido o que é uma expressão regular, para que elas servem e como implementa-las.

\subsubsection{Introducão}
As expressões regulares, ou regex do inglês \emph{regular expression}, são uma ferramenta muito poderoza na computação, utilizadas para processar texto.
De maneira simplificada, uma expressão regular é uma linguagem usada para descrever padrões de caracteres.
A partir da expressão regular, e um texto alvo, usa-se um motor de busca que varre o texto a procura de segmentos para o qual a expressão regular é aceita.
Um exemplo de uso seria uma expressão regular para buscar por todas as menções de hora em um texto, assumindo que o horario tenha um padrão uniforme (ex. HH:MM).
Pode-se criar uma expressão regular para esse formato, e usando uma ferramenta de busca, encontrar todos os strings que atendam o formato definido.

Como dito anteriormente, a regex define uma linguagem para definir padrões de texto.
Normalmente, essas linguagens fazem uso de caracteres especiais para indicar operações.
As operções básica são: concatenação, alternação, repetição e agrupamento.

A regex mais simple é simples é um único caractere não especial, por exemplo "0", uma regex que procura pelo caractere 0.
A operação de concatenção é implicita em uma regex, qualquer dois caracteres não especiais estão concatenados.
Sendo assim, a regex "01" procura pelo string "01".
A alternação normalemente é indicada pelo caractere "|", que indica que um caractere ou outro é valido.
Um exemplo de alternação é a regex "01|0", que procura pelos strings "01" ou "00".
Existem vários operadores de repetição, um dos mais usados é o operador de Kleene, normalmente indicado por "*", esse operador indica que o caractere, ou grupo, que o precede pode ocorrer 0 ou mais veses.
Por exemplo, a regex "01*" é equivalente a "0", "01", "011" e "0111...", ou seja, qualquer string que tenha um "0" seguido por qualquer numero de "1"s.
Por ultimo, o agrupamento é definido usando parenteses "(01)", normalmente utilizado para indicar a repetição de um grupo de caracteres.

Existem várias outras funcionalidades e operadores, porém esses são os básicos.
Para uma referência mais exaustiva, consulte \cite{mastering}.

\subsubsection{Teoria das regex}
As expressões regulares tem origem na teoria das linguagens formais, um assunto muito importante que formalizou a síntaxe das linguagens de programação.
Fora a algebra por traz desse tópico, existem ainda os inúmeras diferentes dialetos para regexes, visto que a existem varias implementações diferentes com funcionalidades distintas.
Sendo assim, regexes são um tópico extenso e complexo, que foje do escopo deste trabalho.
Para tanto, será explicado como uma regex é modelada em um ambiente computacional, o mínimo necessário para serem implementadas.

Uma regex é equivalente a uma máquina de estados, ou automata \cite{theory-computation}.
Uma máquina de estado é um bom modelo mátematico para um computador limitador \cite{theory-computation}.
A máquina de estado opera sobre símbolos de entrada, a cada símbolo enviado à ela, ela muda de estado.
A computação da máquina de estado encerra quando não existem mais símbolos de entrada, caso ela esteja em um estado de aceitação, a computação foi bem sucessida.

Formalmente, uma maquina de estados consiste de: estados, símbolos de entrada, um estado de inicio, estados de aceitação e uma função de transição.
Os estados são denomidados por um nome único, normalmente um número.
Os símbolos de entrada são o conjunto de caracteres que a maquina de estados reconhece.
O estado de inicio é o estado inicial da máquina de estados, sempre que iniciada ela se encontra nesse estado.
A maquina de estado pode ter um conjunto de estados que são considerados válidos quando não existem mais símbolos de entrada.
A funcão de transicão é responsável por inteligar estados, essa função recebe um símbolo de entrada, o estado atual e retorna um novo estado. \cite{theory}

Uma regex é equivalente a uma automata, segundo \cite{dragon-book}, podemos construir uma automata para uma regex de maneira indutiva.
Na literatura, é enumerado as diferentes automatas equivalentes as regex primitivas, junto de como combinar automatas.
Sendo assim, para  construir um motor de busca deve-se converter os primitivos da regex em uma automata primitiva e em seguinda, combinar as automatas.

Em conclusão, as expressões regulares são usadas para buscar padrões de texto.
As expressões regulares são definidas usando uma linguagem própria, onde alguns caracteres tem significado especial.
É possível converter uma regex em uma automata e usando o modelo da automata, é possível realizar uma busca em texto por uma expressão regular.

