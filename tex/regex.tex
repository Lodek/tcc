\section{Referencial Teórico}

\subsection{Expressões Regulares}
As expressões regulares foram escolhidas como o problema computacional de interesse para introduzir programação funcional, porém como elas não são o foco deste trabalho, sua abordagem será simplificada.
Nessa seção será introduzido o que é uma expressão regular, para que elas servem e como implementá-las.

\subsubsection{Introdução}
As expressões regulares, ou regex do inglês \emph{regular expression}, são uma ferramenta muito poderosa para processamento de textos.
De maneira simplificada, uma expressão regular é uma linguagem usada para descrever padrões de caracteres \cite{mastering}.
A partir da expressão regular e um texto alvo, usa-se um motor de busca que varre o texto a procura de segmentos para o qual a expressão regular é aceita.
Um exemplo de uso seria uma expressão regular para buscar por todas as menções de hora em um texto, assumindo que o horário tenha um padrão uniforme (ex. HH:MM).
Pode-se criar uma expressão regular para esse formato, e usando uma ferramenta de busca, encontrar todos os strings que atendam o formato definido.

Como dito anteriormente, a regex define uma linguagem para definir padrões de texto.
Normalmente, essas linguagens fazem uso de caracteres especiais para indicar operações.
As operações básica são: concatenação, alternação, repetição e agrupamento.

A regex mais simple é um único caractere não especial, por exemplo "0", uma regex que procura pelo caractere "0".
A operação de concatenação é implícita em uma regex, qualquer dois caracteres justapostos, não especiais, estão concatenados.
Sendo assim, a regex "01" procura pelo string "01".
A alternação normalmente é indicada pelo caractere "$\|$", que indica que um caractere ou outro é válido.
Um exemplo de alternação é a regex "01|0", que procura pelos strings "01" ou "00".
Existem vários operadores de repetição, um dos mais usados é o operador de Kleene, normalmente indicado por "*", esse operador indica que o caractere, ou grupo, que o precede pode ocorrer 0 ou mais vezes.
Por exemplo, a regex "01*" é equivalente a "0", "01", "011" e "0111...", ou seja, qualquer string que tenha um "0" seguido por qualquer número de "1"s.
Por ultimo, o agrupamento é definido usando parênteses "(01)", normalmente utilizado para indicar a repetição de um grupo de caracteres\cite{mastering}.

Existem várias outras funcionalidades e operadores, porém esses são os básicos.
Para uma referência mais exaustiva, consulte \cite{mastering}.

\subsubsection{Teoria das regex}
As expressões regulares tem origem na teoria das linguagens formais, um assunto muito importante que formalizou a sintaxe das linguagens de programação \cite{theory-computation}.
Fora a álgebra por trás desse tópico, existem ainda os inúmeros diferentes dialetos para regexes, visto que a existem várias implementações diferentes com funcionalidades distintas.
Sendo assim, regexes são um tópico extenso e complexo, e uma abordagem completa foge do escopo deste trabalho.
Contudo, será explicado como uma regex é modelada em um ambiente computacional, o mínimo necessário para serem implementadas.

Uma regex é equivalente a uma máquina de estados, ou automata \cite{theory-computation}.
Uma máquina de estado é um bom modelo matemático para um computador limitado, que opera sobre símbolos de entrada.
A máquina de estado processa cada símbolo de entrada e muda de estado de acordo com sua construção. \cite{theory-computation}.
A computação da máquina de estado encerra quando não existem mais símbolos de entrada, caso ela esteja em um de seus estado de aceitação, a computação foi é válida.

Formalmente, uma máquina de estados consiste de: estados, símbolos de entrada, um estado de início, estados de aceitação e uma função de transição.
Cada estado é denominado por um nome único, normalmente um número.
Os símbolos de entrada são um conjunto de caracteres que a máquina de estados reconhece, o conjunto de todos os símbolos reconhecidos forma o alfabeto da automata.
O estado de início é o estado inicial da máquina de estados, sempre que iniciada ela se encontra nesse estado.
A máquina de estado possui um conjunto de estados de aceitação, caso a computação encerre em um desses estados, a computação foi válida.
A função de transição é responsável por interligar estados, essa função recebe um símbolo de entrada, o estado atual e retorna um novo estado. \cite{theory}

Uma regex é equivalente a uma automata, segundo \cite{dragon-book}, podemos construir uma automata para uma regex de maneira indutiva.
Na literatura, é enumerado as diferentes automatas equivalentes às regex primitivas, junto de como combinar essas automatas.
Sendo assim, para  construir um motor de busca deve-se converter os primitivos da regex em uma automata primitiva e em seguida, combinar as automatas.

Em conclusão, as expressões regulares são usadas para buscar padrões de texto.
As expressões regulares são definidas usando uma linguagem própria, onde alguns caracteres tem significado especial.
É possível converter uma regex em uma automata e usando o modelo da automata, é possível realizar uma busca em texto por uma expressão regular.
