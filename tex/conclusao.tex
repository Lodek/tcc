\section{CONSIDERACOES FINAIS}

O objetivo desse trabalho foi expor alguns conceitos por traz do paradigma funcional e demonstrar como esses conceitos sao uteis atraves da construcao de um modulo de processamento de expressoes regulares.
O modulo foi implementado usando a linguagem Haskell, uma linguagem funcional pura.

Primeiramente foi abordado as origens da programacao funcional, tal como as principais diferencas entre o paradigma funcional e imperativo.
Ainda, foram introduzidos alguns conceitos exclusivos das linguagens funcionais, tais como Monads.
Foi abordado de maneira abrangente o conceito por traz das expressoes regulares (regexes) e alguns casos onde sao uteis, juntamente com parte da sua historia.
Por ultimo foi discutido diferentes metodos para se resolver o problema computacional referente a pesquisa de expressoes regulares.

Usando os conceitos apresentados foi demonstrado o processo de construcao do modulo de regex.



