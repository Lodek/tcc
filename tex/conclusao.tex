\section{CONSIDERAÇÕES FINAIS}

O objetivo deste trabalho foi expor alguns conceitos por trás do paradigma funcional e exemplificar esses conceitos através da construção de um módulo de processamento de expressões regulares.
O módulo foi implementado usando a linguagem Haskell, uma linguagem funcional.

Primeiramente, foi explicado o que são regexes do ponto de vista de um usuário e quais problemas elas resolvem.
Em seguida, foi introduzido os conceitos por trás das expressões regulares, tais como sintaxe e operações.
Juntamente, foi introduzido as máquinas de estado, ou automatas, sua definição e como simular uma automata em um ambiente computacional.
Finalmente, foi visto a equivalência entre uma expressão regular e uma automata.

Sobre o paradigma funcional, comentou-se sobre imutabilidade, \emph{laziness}, funções como cidadãos de primeira classe e funções de ordem superior.
Durante a exposição inicial do paradigma funcional, foi contrastado como as origens desse paradigma difere do paradigma imperativo.

Para a implementação da biblioteca proposta, foi analisada a arquitetura projetada para a ferramenta.
A arquitetura é composta por 4 módulos, que realiza a transformação de uma regex em formato de string para sua automata equivalente.
Para cada módulo, foi explicado o seu fluxo lógico e as funções que o compõe.

Por fim, foram selecionados alguns dos problemas encontrados na implementação da biblioteca e analisamos como resolver esses problemas  fazendo uso dos conceitos introduzidos.
Essa abordagem permitiu expor os pontos as diferenças entre os paradigmas, mostrando como resolver problemas de uma maneira funcional.

Em conclusão, esse trabalho tem o objetivo de mostrar um mundo diferente da programação, um mundo que vem sido incorporado às linguagens imperativas, mesmo que muitos programadores desconhecem suas origens.

