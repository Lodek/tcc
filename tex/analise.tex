\section{Analise e Discussão}

A analise do codigo escrito sera feita por modulo.
Sera abordado topicos relevantes sobre cada modulo a fim de exemplificar conceitos da programacao funcional.
Por fim, sera feita uma analise sobre o processo de desenvolvimento como um todo.

\subsection{Modulo de Parse}

O modulo de parse é reponsável por transformar uma regex em formato de string em uma árvore.
Um ponto crucial desse módulo é ler os caracteres do stream de entrada e identificar quando uma expressão completa foi lida.
Por exemplo, como armazenar os caracteres que compõem uma subexpressão em uma expressão regular.

Em uma linguagem imperativa, esse problema é trivial, basta adicionar os símbolos da subexpressão em uma estrutura de pilha e remove-los quando eles formarem uma unidade completa.
Devido a imutabilidade das linguagens funcionais, isso não é possível

Uma solução que respeita a imutabilidade é usando recurssão.
Quando a função de converssão encontra um caractere que indica o inicio de um grupo, essa função chama a si propria para que seja construída a árvore da subexpressão.
Essa função lê os caracteres de entrada até que seja identificado o fim de um grupo, onde é retornado a árvore para a subexpressão.
Essa solução dispensa o uso de uma pilha e flags para guardar o estado do código, respeitando assim a imutabilidade do código.

Com as árvores construidas elas dever ser unidas usando operadores, onde para cada operador, as duas primeiras árvores da pilha de árvores são unidas e o resultado é adicionado ao topo da pilha.
Nessa caso, a solução requer tanto mutação quanto repetição.
Esse problema pode ser resolvido usando um fold, basta notar deseja-se consumir o valor de uma lista e gerar um único valor a partir disso.
A função de união itera sobre os operadores e usa como acumulador a lista de árvores.
Para cada operador, as duas primeiras árvores são consumidas e o novo acumulador é uma nova lista onde a árvore resultante é o primeiro elemento, concatenado aos elemntos restantes da lista original.
No final da execução o acumulador é uma lista com um unico elemento, a árvore final.

\subsection{Modulo de Automatas}
O modulo de automata implementa algoritimos que permite a simulação de uma automata.

Seguindo a literatura, uma das funções usadas para simular uma automata deve receber um conjunto de estados e retornar um conjunto de todos os estados acessiveis usando uma transição nula, a função "epsilon closure".
A soluçã̀o desse problema imperativamente faz uso de uma estrutura de repetição, uma lista de resultados a ser visitados, e uma lista de estados visitados.
Para cada estado a ser visitado, adicione-o a lista de estado visitados e teste quais estados são acessíveis a partir dele, para cada estado acessível, cheque se ele está na lista de estados visitados, caso não, adicione-os á lista de resultados a ser visitados; esse procedimento é repetido até a lista de estados para visitar ficar vazia.

Uma alternativa funcional desse problema é interessante pois demonstra uma solução inusitada, que faz uso de operações de conjunto.
Para essa solução é necessário dois conjuntos, um de estados visitados e outro de estados a ser visitado.
Assim, verifica-se todos os estados acessiveis a partir dos estados de entrada, esse gera um conjunto resultado temporario.
O conjunto de estados a ser visitado passa a ser a diferença do conjunto de estados visitado e o conjunto temporario.
O conjunto de estado visitados passa a ser a união do conjunto original com o conjunto temporario.
A função que gera o conjunto de estados acessível é recursiva e se repete até que os estados a serem visitados forma o conjunto vazio.

A partir da função acima, o resto da implementação da automata é trivial pois as funções da literatura podem ser facilmente traduzidas de maneira funcional.

\subsection{Modulo de conversão}
O módulo de conversão é responsável por transformar uma árvore de parse em uma automata.

O "cerébro" da automata é dada pela função de transferencia, que retorna as possíveis transições para um estado da automata.
Como foi visto, a construção da autmoata é feita de maneira incremental, onde automatas mais complexas são construidas a partir de um conjunto finito de estruturas bases.
Para combinar as funçõ̀es de transição foram definidas funções que recebem duas funções de transiçao e retornam uma nova função de transição.
Esse uso é um exemplo de uma função de ordem superior.

Como funções são cidadoes de primeira classe em Haskell, pode-se passar funções para funções de maneira transparente.
Isso faz com que combinar funções seja feito de maneira trivial, usando uma função lambda como o retorno das funções de combinação.

A partir das fábricas para as operações primitivas, basta converter os nós e as folhas da árvore em automatas.
Novamente, podemos ver um caso de uso para um fold, visto que deseja-se iterar sobre a árvore de parse e usar um único valor como resultado dessa iteração.

Os desafios impostos pela implementação do modulo de conversão foram ideiais para programação funcional.
O problema de combinar funções é feito de maneira clara e concisa devido a natureza do paradigma funcional.
O uso de um fold e pattern matching permite identificar qual tipo de automata deve ser construída para diferentes árvores seja feita de maneira idiomatica.

\subsection{Observações gerais}

O desenvolvimento do codigo proposto usando o paradigma funcional foi desafiador.
A experiencia de um programador imperativo, em muitos casos, gera atrito com o paradigma funcional.
O desenvolvimento eh analogo a sensacao de nao saber qual palavra usar em uma frase.
O modelo mental do que deve ser feito no codigo eh conceitualmente claro porem a implementacao nao eh fluida.

Haskell, sendo uma linguagem puramente funcional, retira as ferramentas normalmente conhecidas de um programador imperativo.
Muitas vezes ao escrever um programa imperativo, o objetivo eh claro e embora o caminho nao esteja totalmente claro, a liberdade dada pela mutabilidade de variaveis e pelo sequenciamento de instrucoes para realizar a computacao permite que o problema seja resolvido de maneira interativa.
Em Haskell, dada as construcoes da linguagem, quanto maior uma funcao mais complicado eh o processo de desenvolver "interativamente".
O programador eh incentivado a usar funcoes pequena para resolver o problema.
O resultado eh um processo diferente para o desenvolvimento de codigo, os passos para resolver um problema em uma linguagem funcional devem estar muito mais claros.
O problema a ser resolvido deve estar totalmente compreendido antes do desenvolvimento comecar, pois visto a dificuldade que eh criar funcoes complexas, todas as funcoes auxiliares e a sequencia de transformacoes sobre o valor de entrada deve estar completamente entendido pelo programador.
Isso nao quer dizer que esse processo eh exato, erros ainda ocorrem, mas o paradigma funcional forca o entendimento do problema de maneira macro antes da implementacao no micro, o que nao eh verdade na programacao imperativa.

Um aspecto que nao ficou claro durante o desenvolvimento do projeto eh como lidar com um problema estruturalmente complexo.
O problema de implementar um motor de busca eh regex tem poucos tipos de dados, somente a arvore de parse e a automata.
Seria interessante ver como uma linguagem funcional se comportaria em um problema com varios tipos de dados e estruturas diferentes.

A implementacao do modulo de parse foi a parte mais desafiador do ponto de vista tecnico.
Os algoritimos bem documentados para parse de um stream de caracteres sao imperativos e nao possuem uma traducao direta para funcoes devido a mutacao.

A construção de testes unitarios em linguagens funcionais pura é um processo mais simples do que em linguagem orientadas a objetos.
Visto que objetos normalmente armazenam algum estado e possui referencia para outros objetos, a construção de testes unitarios involve tecnicas como mocking.
Os mocks são usados para imitar algum tipo de objeto, isso permite que uma classe seja testada de maneira isolada, sem dependencias externas.
O uso de mock resolve o problema porém é um processo tedioso e verboso, sendo necessário especificar o comportamento de vários mocks para cenarios diferentes.
Como em uma linguagem puramente funcional, nenhuma função depende de estado, validar o comportamento de uma função é trivial pois toda função possui uma entrada e uma saída clara.

