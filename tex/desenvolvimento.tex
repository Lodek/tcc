\section{EXPRESSÕES REGULARES E PROGRAMAÇÃO FUNCIONAL}
Como foi abordado na introdução, esse trabalho une dois temas: expressões regulares e programação funcional.
Esses temas serão, primeiramente, discutidos separadamente, e em seguida será feito uma previa de como será feita a construção do motor de processamento de expressões regulares.

\subsection{Introdução as expressões regulares}

Expressões regulares, tambem conhecidas como \emph{regex} (da junção do nome em inglês, \emph{regular expression}) são utilizadas para realizar buscas complexas sobre strings.
Para Cox,
"expressões regulares são uma notação que descreve um conjunto de strings.
Quando alguma string está no conjunto associado à expressão regular, pode-se dizer que essa expressão regular corresponde a esse string." \cite{cox}.

As regexes são utilizadas frequentemente, tanto para extrair informações que seguem um padrão ou para realizar buscas mais flexíveis ou parciais.
Como exemplo, suponha o problema de extrair todas os strings que correspondem a um horario em um texto.
A escrita de um horario segue uma estrutura padrão, HH:MM:SS onde HH delimita as horas, MM delimita os minutos e SS delimita os segundos.
Sem ter que construir todas as possíveis combinações de horas que seguem esse formato, uma simples varredura de texto é incapaz de extrair essa informação.
Esse problema pode ser resolvido tranquilamente usando regexes.

Uma expressão regular que realiza esta busca é,
\begin{equation}
  [0-9]\{2\}:[0-9]\{2\}:[0-9]\{2\} .
\end{equation}

Em palavras, os símbolos [0-9] representa qualquer carácter numerico entre 0 e 9.
De maneira geral os símbolos [] representam um conunto de caracteres \cite{python-re}.
O token \{2\} indica uma repetição, sendo equivalente à regex [0-9][0-9], ou seja dois carácteres numericos.
Os simbolos \{\} são usados para representar repetição \cite{python-re}.
O caractere ":" é interpretado de maneira literal.
Fazendo a união, a regex acima equivale a qualquer string que tenha o formato DD:DD:DD onde D indica qualquer digito de 0-9.
Podemos ver que esse formato é exatamente o formato definido anteriormente.

É importante ressaltar que existem inumeras variações e implementações de regexes, onde existem diferentes meta-caracteres para descrever operações.
Em \cite{mastering}, o autor discute as diferenças em regex entre as linguagens: PHP, .NET, Java e Perl.
Na documentação oficial da linguagem Python \cite{python-re} é dito que o dialeto usado é basedo nas expressões regulares da linguagem Perl com alguns adicionais.
Usuários UNIX também estão familiarizados com os \emph{wildcards} presentes nos shells, uma forma de regex.
Em resumo, existem diversos dialetos porém o objetivo das regexes não se altera, buscar por padrões.
A regex acima e todas as regexes subsequentes nesse texto serão escritos no dialeto da linguagem Perl. 

\subsection{Programação Funcional}

Essa seção aborda programação funcional e suas caracteristicas de maneira resumida.
Há muito a se falar sobre esse assunto pois ele é extensso e tem uma longa história.
Para abordar a programação funcional será tomado um foco que toma como base a computação.

De maneira geral, programas de computadores existem para resolver problemas computacionais.
Segundo \cite{matrix} um problema computacional é
" [...] uma especificação de entrada-saída que um procedimento tenha que satisfazer."
e um procedimento é
"[...] uma descrição precisa de uma computação; ele aceita entradas (chamadas de argumentos) e produz uma saída (chamado de valor de retorno).".
Ou seja, independente do paradigma utilizado para resolver o problema (funcional ou imperativa), ambos são capazes de definir um procedimento para um problema computacional, a grande diferença está em como esse procedimento é definido.

Segundo \cite{Bird},
"programção funcional é: um método para construção de programas que enfatiza funções e suas aplicações ao invéz de commandos e suas execuções; programação funcional faz uso de notações matemática simples que permite que problemas sejam descritos de maneira clara e concisa. [...]".
A programação imperativa foca em passos para resolver um problema, cada passo desse pode ser traduzido de maneira rasoavelmente direta em instruções de uma CPU.
Isso faz com que o procedimeto escrito imperativemente reflita muito mais a máquina do que ao homem.
O paradigma funcional tira o foco nos passos individuais para solucionar o problema e enfatiza uma estrutura para resolver o problema.

Em seguida serão abordados aspectos mais técnicos da programação funcional.

\subsubsection{Funções para resolver problemas}

Como visto anteriormente, a programação funcional propõe que problemas computacionais sejam resolvidos de maneira mais declarativa.
O foco muda de "quais passos é preciso para resolver esse problema" para "quais transformações aplicar nas minhas entradas para produzir a saída".

Para exemplificar essa idea considere o seguinte problema.
Dado uma lista de nomes, com nome, nome do meio e sobrenomes, crie uma lista com todas as combinações de primeiro nome e ultímo nome, ignorando nomes do meio e aonde as combinações cuja soma do primeiro e ultimo nome não exceda 15 caracteres (incluindo o espaço).
Para isso, ao invez de analisar os passos para processar esses dados, uma boa ideia é pensar em como manipular os dados para se obter o resultado desejado.
Para esse problema, sugere-se a seguinte solução:

\begin{enumerate}
\item{Separar cada nome da lista de nomes nos espaços e armazenar os nomes uma lista.}
\item{Filtrar listas que só possuem um elemento (somente um nome).}
\item{Filtrar listas e remover nomes do meio.}
\item{Criar uma lista de nomes e uma lista sobrenomes.}
\item{Realizar o produto cartesiano sobre essa lista e gerar uma lista de tuplas.}
\item{Tranformar tuplas em strings fazendo a concatenação do primeiro nome e do sobrenome.}
\end{enumerate}

Percebe-se que cada passo acima realiza uma única ação, sendo ela simples e clara e é interessante modelar cada um desses passos como uma função.
Na linguagem Haskell o tipos dos argumentos e do retorno de uma função é dado pela notação nomeDaFuncao :: arg1 -> arg2 -> ... -> retorno, onde arg1 e arg2 definem os tipos dos argumentos \cite{lipovaca}.
Para identificar listas em Haskell é usado o símbolo [], ou seja [Char] indica uma lista de caracteres e tuplas são indicatas com () onde (String, String) indica uma tupla com dois elementos, ambos strings.
Podemos agora reescrever o problema acima definindo todas as funções que serão utilizadas.

Primeiramente definiremos o problema enunciado como uma função usando a notação introduzida.
O problema inicial é a função $combinarNomes :: [String] -> [String]$ , ou seja uma função que recebe uma lista de Strings e retorna uma lista de Strings.
Em seguida, iremos declarar as funções que representam cada passo acima.

\begin{enumerate}
\item{separarNomes :: [String] -> [[String]]}
\item{tirarIncompleto :: [[String]] -> [[String]]}
\item{removerSobrenomes :: [[String]] -> [[String]]}
\item{gerarNomesESobrenomes :: [[String]] -> ([String], [String])}
\item{gerarCombinacoes :: ([String], [String]) -> [(String, String)]}
\item{concatenarNomes :: [(String, String)] -> [String]}
\end{enumerate}

Segundo \cite{lipovaca} a assinatura de uma função em Haskell combinado com um nome descritivo diz muito sobre a função e de fato, dados os nomes e sua assinatura, pode-se facilmente deduzir o que cada função está fazendo.

O objetivo dessa seção foi dar um exemplo alto nível de como é resolvido um problema de maneira funcional.

\subsection{Automatas e expressões regulares}

Como visto, as expressões regulares representam uma maneira conveniente de descrever conjutos de string.
Embora conveniente, a maneira na qual as expressões regulares foram introduzidas não permite uma tradução direta delas para um ambiente computacional.
Essa seção faz a ligação entre esses objetos teóricos e uma descrição matématica das mesmas.

\subsubsection{Definição de uma automata}

Segundo \cite{comp}, as automatas modelam um computador com uma quantidade minúscula de memória.
A ideia central de uma automata é representar uma estrutura computacional a partir de um conjunto de estados e entradas.

Os estados da automata constitui um conjunto denominado de $Q$, o conjunto de estados.
Dentre esses estados existe um único estado inicial da automata chamado de $q_o \in Q$.
Automatas recebem entradas a partir de simbolos, o conjuto de todos os símbolos reconhecidos por uma automata define um conjunto $\Sigma$ chamado de alfabeto da automata.
Os estados da automata podem ou não estar conectados, quando existe uma conexão entre dois estados essa conexão é representada por um simbolo $\alpha \in \Sigma$.
As transições entre estados de uma automata é representado por uma função $\delta$ onde $\delta : Q \cross \Sigma \mapsto Q$, ou seja $\delta$ recebe dois argumentos, um estado e um símbolo e mapeia esse par a um estado.
Finalmente, a automata possui um conjunto de estados de aceitação $F$, onde $F \in Q$, caso a automata termine sua execução em um estado $q \in F$, o string de entrada foi aceito pela automata.
Formalmente, então, uma automata é uma tupla com 5 elementos $(Q, \Sigma, \delta, q_o, F)$ \cite{comp}.

A partir da descrição formal de uma automata, podemos definir uma rotina de computação.
De maneira breve, o objetivo dessa rotina é verificar que após processar o string de entrada a automata se encontra em um estado de aceitação.

Como foi visto, uma automata pode receber um conjunto de entradas que definem seu alfabeto $\Sigma$.
Deseja-se definir um procedimento onde dado um string de entrada e uma automata no seu estado inicial, retorne o estado final após processar a entrada.
Esse procedimento é definido como dado uma entrada $w=w_1w_2...w_n \suchthat w_i \in \Sigma$ e uma automata $M$ no seu estado inicial, será retornado um estado $q \in Q$.
Caso o estado final seja um estado de aceitação ($q \in F$), é dito que $M$ aceita $w$ \cite{comp}.
Formalmente, segundo \cite{comp} $M$ aceita $w = w_1w_2...w_n$ se: existe uma sequencia de estados $r_0, r_1, ... r_n \in Q$ se $r_0 = q_0$; $\delta(r_i, w_{i+1}) = r_{i+1},$ para $i = 0, ..., n-1$; $r_n \in F$.

O ponto chave desse discussão é apresentado por \cite{comp}, onde foi provado que é possível construir uma automata para qualquer regex.
Sendo assim, é possível definir padrões de busca usando uma expressão regular, converter essa expressão regulara para uma automata e usar essa automata para realizar a busca pelo padrão em um string.

\section{METODOLOGIA}

%we doing it live 
%tools used to generate the data
%what is needed to reporduce the work

Nessa seção será abordada a arquitetura do software desenvolvida, isso permite uma visão holistica que define uma estrutura.
O software desenvolvido é uma biblioteca para busca usando expressão regular.
Essa biblioteca foi quebrada em três modulos publicos e um privado.
Os modulo internos definem os tipos de dados e implementam as transformações, enquanto o modulo público serfve como uma interface que conecta os modulos e uma interface de entrada / saida.
Os quatro modulos são: modulo de parse, modulo de automata, modulo de conversão e modulo publico.

\subsection{Modulo de Parse}

O modulo de parse é o primeiro estágio da pipeline que irá permitir a construção de uma automata de busca.
Esse módulo é responsável por converter a entrada do usuário (uma String) em uma estrutura intermediária que é consumida pelo módulo conversor.

A saída do parser léxico é uma estrutura em árvore, similar à uma árvore de parse gerada por um compilador.
Nessa árvore, as folhas dela são as primitivas do alfabeto de entrada (letras como “a”, “b” ou “c”), e os nós se interligam através de operadores, como os operadores de concatenação e alteração da regex.
Existe uma exceção para os nós pois eles podem representar uma quantificação também.
Logo, um nó ou é uma operação que liga subárvores ou é uma quantificação que permite o uso do operador “*”, por exemplo.

A escolha de uma árvore para essa estrutura intermediária é conveniente por dois motivos: subárvores apresentam uma tradução, quase que, direta com as automatas primitivas que equivalem a expressões regulares e o uso da árvore elimina as ambiguidades referentes à ordem das operações, sem precisar fazer uso de parênteses para indicar a qual grupo de caracteres um operador opera sobre.

Para construir a árvore, o modulo define alguns tipos de dados.
Na fígura \ref{f-parse-data} é dada a definição dos tipos de dados definidos.
O típo \emph{Symbol} é sinomimo de um tipo de caractere.
O tipo \emph{Operator} define uma enumeração, podendo ser ou uma concatenação ou uma alternação.
O tipo \emph{Quantifier} define uma enumeração, representando o operador de Kleene, também conhecido como estrela.
O tipo \emph{Token} define um grupo de construções, podendo ser um Token simbolico, token de quantificação, token de operação, delimitador de inicio de grupo ou delimitador de fim de grupo.
O tipo \emph{SubEpression} representa uma sub-expressão composta por uma lista de tokens ou uma subexpressão quantificada.
Finalmente, o tipo \emph{ParseTree} representa uma árvore, podendo ter uma folha contendo um símbolo; um nó contendo uma arvore e uma quantificação; um nó contendo uma árvore, operador e outra árvore.


A implementação desse módulo em Haskell foi feita usando um conjunto de funções que opera sobre os tipos definidos anteriormente.
A tabela \ref{t-parse-funcs} indica as funções desse módulo, junto com seus tipos de entrada, saída e uma breve descrição sobre cada uma.
Note que a função buildTree recebe uma String e retorna uma árvore, sendo assim essa função realiza a transformação completa sobre a entrada.

\begin{table}
  \begin{tabular}{lll}
  \hline
  Nome & Assinatura & Descrição \\
  \hline
  genToken & Symbol -> Token & Transforma um símbolo em um token. \\
  genTokens & [Symbol] -> [Token] & Transforma uma lista de símbolos em uma lista de tokens. \\
  normalizeStream & [Token] -> [Token] & Adiciona operador de concatenação entre símbolos justapostos. \\
  evenGroupPredicate & [Token] -> Bool & Valida que existe numero par de parenteses em uma regex. \\
  uniqueQuantifierPredicate & [Token] -> Bool & Valida que não existem quantificadores justapostos. \\
  yankSubExp & [Token] -> (SubExpression, [Token]) & Extrai uma subexpressão do stream de tokens. \\
  takeWhileGroupUneven & [Token] -> [Token] & Remove tokens até que um grupo completo seja formado. \\
  takeWhileList & ([Token], Bool) -> [Token] -> [Token] -> [Token] & Remove items da primeira lista e adiciona a segunda enquanto o predicado é verdadeiro.\\
  validateTokens  & [Token] -> Bool & Valida tokens \\
  sortAndTreefy & [Token] -> [Either Operator ParseTree] & Classifica Token como sendo um Operador ou uma árvore. \\
  buildSupExp & SubExpression -> ParseTree & Transforma uma subexpressão em uma arvore. \\
  transformEithers & [Either Operator ParseTree] -> ([Operator], [ParseTree]) & Agrupa operadores e árvores. \\
  mergeOps & [Operator] -> [ParseTree] -> [ParseTree] & Combina árvores usando uma lista de operadores. \\
  buildTree & String -> ParseTree & Transforma um string em uma árvore. \\
 \hline
  \end{tabular}
\caption{Tabela de funções para o modulo de parse. Cada função é apresentada com sua assinatura e uma breve descrição.}
\label{t-parse-funcs}
\end{table}


A arquitetura do módulo de parse é dada pelas funções na tabela \ref{t-parse-funcs} e os tipos de dados enumerados no texto.
A partir desses elementos, é possível implementar as funções tal que o modulo receba uma expressão regular em uma string e retorne uma árvore que preserve o significado semantico da expressão regular de entrada.

\subsection{Modulo de Automata}

O modulo de Automata implementa o modelo computacional das maquinas de estados.
Esse modulo define um tipo de dado que representa uma maquina de estados e expõe funções que operam sobre uma maquina de estado.

Primeiramente, foram definidos os tipos de dados necessarios para modelar a maquina de estado.
A fígura \ref{f-automata-data} contém o trecho de código que define os tipos de dados.
Foi definido o tipo \emph{SigmaElem} que representa um elemento do alfabeto da automata.
O tipo \emph{Sigma} o alfabeto da automata, ou seja conjunto de \emph{SigmaElem}.
O tipo \emph{State} é um alias para um numero inteiro.
O tipo Delta é um alias para função de transição da maquina de estado, igual ao definido na literatura.
Finalmente, analogo à literatura, uma maquina de estados é uma tupla de cinco elementos: um alfabeto de tipo \emph{Sigma}, um conjunto de estados, o estado inicial, conjunto de estados de aceitação e a função de transição.
Percebe-se que a modelagem de uma automata segue exatamente o modelo definido na literatura.

As funções definidas nesse módulo são apresentadas na tabela \ref{t-automata-funcs}.
A implementação da automata foi feita de acordo com a literatura \cite{dragon-book}, onde é apresentado os algoritimos para simular uma automata.
O algoritimo apresentado define algumas funções auxiliares, tal como \emph{epsilonClosure} e \emph{move}.
As outras funções do modulo foram introduzidas devido a conversão do algoritimo imperativo da literatura para um algoritimo funcional.

Usando a arquitetura apresentada, a implementação das funções definidas permite a simulação de uma maquina de estados.
Essa maquina de estados sera utilizada para executar a busca pela expressão regular.
Para isso, é necessário converter a regex em árvore em uma maquina de estados.

\subsection{Modulo de conversão}

O modulo de conversão é responsável por converter a expressão regular armazenada em uma árvore em uma automata que possa ser executada.

A literatura contém maquinas de estados para as primitivas de ume expressão regular.
Usando essas primitivas, é possível combina-las para construir expressões mais complexas.
O módulo de conversão faz disponibiliza funções para criar e combinar essas primitivas.
A partir dessas funções, basta trafegar pela árvore para transforma-la em um unico valor, ou seja, usar um \emph{fold}.

A tabela \ref{t-translator-funcs} apresenta as funções definidas para o módulo.
Note que esse modulo não introduz um tipo de dado próprio, apenas realiza transformações.

\subsection{Modulo público e comando cli}
O módulo público cria uma interface para que a biblioteca seja facil de utilizar.
Nesse módulo são definidas funções auxiliares e também um programa que pode ser rodado de maneira idependente.

As funções do modulo público são apresentadas na tabela \ref{t-wrapper-funcs}.
Essas funções apenas fazem chamadas para as funções dos diferentes modulos da biblioteca e não introduz nenhuma logica nova.

O comando CLI é diferente pois ele deve tratar com a entrada e saída de valores, ou seja, realizar operações com o sistema.
O programa criado recebe uma expressão regular como um argument do programa e le um texto para receber realizar a busca no stream de entrada padrão (stdin) e retorna o trecho do texto que satisfaz a regex.

\subsection{Práticas comuns e conclusão}
Embora os módulos da biblioteca tenham objetivos distintos, algumas práticas foram utilizadas para implementar todos eles.
Para o desenvolvimento das funções foi feito o uso de uma metodologia usando testes untiarios, em que para cada função é escrito um teste para validar seu comportamento.
Para isso, foi utilizado a biblioteca HUnit do Haskell, uma biblioteca que auxilia na execução de casos de testes.
A prática de desenvolver testes para as funções dos módulos foi de extrema importância pois permite validar cada unidade lógica do código de maneira individual.

Nessa seção foram introduzidos os diferentes módulos que compõe a biblioteca proposta: parse, automata, conversão e publico; também foi visto as as funções propostas para cada módulo.
Essa descrição da arquitetura do software permite o seu entendimento sem entrar nas nuancias associadas a implementação das funções.
Além disso, permite que diferentes implementações existam, pois uma vez definida a interface do programa, basta que as partes previstas existam para que o mesmo funcione, sem existir a dependencia de como a interface é implementada.
O código completo para a biblioteca está disponivel no apendice 1 e no site http://github.com/lodek/regex-engine.

